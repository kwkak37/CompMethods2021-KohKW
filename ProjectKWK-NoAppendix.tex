\documentclass[ProjectKWK]{subfiles}
% WARNING: AuCTeX local variables only get reset when file is loaded
% and differ between this file and ProjectKWK.tex
% so must re-load whichever file you want to compile with C-x C-v

% WARNING: Different AucTeX execution depending on whether
% 0. Being compiled as standalone document
%    * Compile main once
%    * Then compile this one
%    * Keep compiling until nothing changes
% 0. Being compiled as subfile of main document
%    * Just compile main document repeatedly

\input{./econtexRoot}
\input{\LaTeXInputs/econtex_onlyinsubfile}
\onlyinsubfile{\externaldocument{ProjectKWK}} % Get xrefs -- esp to appendix -- from main file; only works properly if main file has already been compiled;

\begin{document}

% Attempted to make all lines used for Web version contain {Web} (or version with only single curly brace at end) so can be removed with sed
\providecommand{\versn}{pdf} % Version; like, web or pdf or journal submission
\ifthenelse{\boolean{Web}}{    % {Web}
  \renewcommand{\versn}{Web}     % Too hard to figure out passing -output-directory through make4ht through htlatex, so web version is compiled with junk files in main directory
  \renewcommand{\rootFromOut}{.} % {Web}
}{}  % {Web}

% Tiny info header at top to track git commit
%\hfill{\tiny \jobname~\versn~\today~{at} \DTMcurrenttime, \input{\ResourcesDir/.git-source-commit}~~\input{\ResourcesDir/.git-public-commit}}

\title{Close Elections, Campaign Contributions,\\ and Financial Deregulation}

\author{Kyung Woong Koh\authNum}

\keywords{Campaign Contributions, Close Elections, Financial Deregulation, Global Financial Crisis}

\jelclass{D72 \par
  \href{https://econ-ark.org}{\includegraphics{\ResourcesDir/PoweredByEconARK}}
}

\renewcommand{\forcedate}{December 1, 2021}\date{\forcedate}
\maketitle
\hypertarget{abstract}{}
\begin{abstract}
This paper builds upon \cite{IM14} on vote switches towards financial deregulation by US legislators. I measure the effect of close elections on US legislators on switching their votes towards financial deregulation in Congress bills. I aim to distinguish between vote switches towards financial deregulation because of voters' general interests (especially after the Global Financial Crisis of 2007) versus the financial industry's special interests and the industry's campaign contributions and lobbying expenditures towards legislators in close elections.
\end{abstract}


\begin{authorsinfo}
  \name{Contact: \href{mailto:kkoh8@jhu.edu}{\texttt{kkoh8@jhu.edu}}, Department of Economics, Johns Hopkins University, Baltimore, MD 21218.}
\end{authorsinfo}

\newcommand{\thankstext}{
  The paper is the author's 2nd year paper for fulfillment to the PhD in Economics program at Johns Hopkins University, and also for Professor Christpher Carroll's Computational Methods course. I thank Professors Laurence Ball and Filipe Campante for comments and advice, and Deniz Igan and Prachi Mishra for sharing their data.}

\ifthenelse{\boolean{Web}}{}{
  \begin{minipage}{0.9\textwidth}
    \tiny \thankstext
\end{minipage}
} % {Web}
{\titlepagefinish}

\hypertarget{Introduction}{}
\section{Introduction}\label{sec:intro}


\begin{figure}[ht]
  \centerline{
    \includegraphics[width=0.8\textwidth]{\FigDir/KohFigure1_tikzmake}
  }
    \caption{Two Channels in which Legislators in Close Elections Switch their Votes Toward Financial Deregulation}\label{KohFigure1}
\end{figure}
\hypertarget{TestFig1}{}
\begin{figure}
\centerline{\includegraphics[width=6in]{\FigDir/TestFig1}}
\caption{Histogram of Legislators' Past Margin of Victory}
\label{fig:TestFig1}
\end{figure}


How do legislators in close elections vote differently on financial deregulation compared to legislators in sure elections? I use data from \cite{IM14} to answer this question.


\hypertarget{The-Problem}{}

\section{The Problem}

\indent I aim to answer part of a wider question of whether legislators, in their legislative activities, are influenced by lobbying and campaign contributions from private interests. Take the Finance, Insurance, and Real Estate (FIRE) industry as an example. As FIRE firms wish to increase the amount of loans that they can make to consumers, they may try to influence legislators to change their stances (and hence votes) towards financial deregulation that allows more lending activities. An ideal study would have multiple bills of the same type of financial regulation, such that we can identify which legislators switch their votes or co-sponsorship towards financial deregulation. In the case of votes, a legislator would initially vote against financial deregulation in an earlier reincarnation $R$ of a bill group $B$, then vote for financial deregulation in a later reincarnation of the same bill group $B$. In the case of co-sponsorship, a legislator would initially not co-sponsor an earlier reincarnation, then co-sponsor a later reincarnation.\\

\indent \cite{IM14} look at the case of U.S. Congressional legislators from 1999 to 2006. To do so, they focus on aggregate lobbying expenditures by the FIRE industry on six groups ($B = 1, 2, \dots, 6$, in total composed of 47 bills. They set as their dependent variable a dummy variable on whether the Congressional legislators' voting records or bill co-sponsorship status \textit{switched} towards financial deregulation (i.e. easing lending activities to consumers). \\

\cite{IM14} argue that their results, that lobbying expenditures make legislators more likely to switch their votes towards financial deregulation, is a correlational result and therefore cannot be taken as causation from either of these reasons. I attempt to provide at least a partial answer by introducing new variables, including one for whether a legislator has been in or will be facing a close election. I also include other variables to support the regression strategies that can differentiate between voters' public scrutiny and the financial industry's special interests.

\indent The next step is to identify variation in the general interest of voters on financial (de)regulation. There are several possible ways of doing this.
\begin{enumerate}
    \item I would extend the existing dataset of \cite{IM14}, from 1999 to 2006, to the Global Financial Crisis (GFC) starting in 2007 and beyond. Given the perceived increase in voters' general interests against financial deregulation after the GFC, the changes in general interest allow me to identify their effect on legislators switching their vote towards financial deregulation, and thus the effect of special interests on legislators' votes as well.
    \item I would include a measure of ``media congruence'' as named by Snyder and Stromberg (2010). Media congruence is defined as the geographical alignment between local news markets and congressional districts. Intuitively, if a congressional district is more geographically aligned with the local news markets, then the legislator representing that district may be covered by the local newspapers more closely than otherwise. The greater public scrutiny incentivizes legislators to vote more on behalf of his/her district voters and less on behalf of special interests such as the FIRE industry. Snyder and Stromberg (2010) include a dataset with an index of media congruence for all Congressional districts from 1984 to 2006.
\end{enumerate}





\subsection{Setup}\label{subsec:Setup}





\hypertarget{Estimation Strategy}{}

\section{Variables}

\subsection{Dependent Variable}\label{subsec:DependentVariable}

\newsavebox{\DepVar}
\begin{table}
\centering
\caption{Definition of the Main Dependent Variable, Vote Switch towards Deregulation}\label{table:DepVar}
\sbox{\DepVar}{
\begin{tabular}{p{0.25\linewidth} | p{0.25\linewidth} | p{0.25\linewidth}} \hline
\textbf{Value of $S_{iBR}$}  & Voted for deregulation in Bill $B,R$ & Voted against deregulation in Bill $B,R$ \\ \hline
Voted for deregulation in Bill $B,R-1$ & 0 & 0 \\ \hline
Voted for deregulation in Bill $B,R-1$ & 1 & 0 \\ \hline
\end{tabular}
}
\usebox{\DepVar}
\end{table}

\subsection{Election Closeness}\label{subsec:ElectionCloseness}

Compared to the original \cite{IM14} paper, I add the new variable of ``Election Closeness'' as the main focus of this paper. I define ``election closeness'' for each legislator $i$ and bill $BR$, denoted as $X_{iBR}$, as the degree to which the legislator has faced (past) or will face (future) a close election. Note the two possible ways of looking at election closeness: through the past election(s) of the legislator, or to the (immediately next) future election of the legislator. Meanwhile, the measure of future election closeness
\\

Past election closeness is defined simply, in this paper, as the percentage margin of victory of the legislator in the last election before a vote on a bill $BR$. I denote this variable as $X^{P}_{iBR}$. As a concrete example, assume that legislator $A$ won his/her last election with 49,000 votes against the runner-up, who got 47,000 votes in a congressional election of a total of 100,000 votes, and that the remaining 4,000 votes all went to third-party, independent, and write-in votes. In this case, legislator $A$'s margin of victory is $(49,000-47,000)/100,000 = 2\%$. There are two important characteristics of this variable: first, $X^{P}_{iBR}$ must necessarily be greater than zero, as the legislator must have won at least one more vote than the runner-up. Second, $X^{P}_{iBR}$ is the same across all bills $BR$ during the same congress $C$, as legislator $i$ has only one value of past election closeness at any given congress $C$ and all the bills therein. \\

Ideally, future election closeness may be defined as the expected margin of victory in the next (future) election of the legislator. I denote this variable as $X^{F}_{iBR}$. The future expected margin of victory of an as-of-yet undecided election can be proxied by results of election polls. Since most future congressional elections at any given electoral cycle have at least one polling result, and in most cases more than one, $X^{F}_{iBR}$ can differ across bills $BR$ even in the same congress $C$.


\subsection{Original Variables}\label{subsec:OriginalVariables}



\hypertarget{Regressions}{}

\section{Regressions}

\subsection{Regression A}\label{subsec:Regression A}


Concretely, I write Regression A1 as:

\begin{align}
          \vLevBF_{T-n} & = \max~ \Ex_{t}\left[\sum_{i=0}^{n} \DiscFac^{i} \uFunc(\cLevBF_{t+i})\right] \label{eq:regA1}
\end{align}



\hypertarget{Results}{}
\section{Results}

See results in Appendix.


\hypertarget{Conclusions}{}
\section{Conclusions}


% %The paper's results are all easily reproducible \href{https://econ-ark.org/_materials/BufferStockTheory?launch}{interactively on the web} or \href{https://github.com/econ-ark/BufferStockTheory}{on any standard computer system}.  Such reproducibility reflects the paper's use of the open-source \href{https://econ-ark.org}{Econ-ARK} toolkit, which is used to generate all of the quantitative results of the paper, and which integrally incorporates all of the analytical insights of the paper.

% % The Dummy equation below sems to be needed to get the equation numbering in the appendix
% % reliably to start at the next number after the last actual equation number used in the paper

% \clearpage\vfill\eject
% \begin{equation*}
%   \label{eq:Dummy}
% \end{equation*}

% \onlyinsubfile{\bibliography{
%     \texname, % subfile inherits texname from preamble of parent
%     \econtexBib % Default bib database is in Resources/LaTeXInputs
%   }}

\onlyinsubfile{\input{\LaTeXInputs/bibliography_blend}}
%\bibliography{economics}
\end{document}
\endinput

% If you are editing in Emacs-AucTeX, modify the lines below for your system (otherwise ignore)
% Local Variables:
% eval: (setq TeX-command-list  (assq-delete-all (car (assoc "BibTeX" TeX-command-list)) TeX-command-list))
% eval: (setq TeX-command-list  (assq-delete-all (car (assoc "BibTeX" TeX-command-list)) TeX-command-list))
% eval: (setq TeX-command-list  (assq-delete-all (car (assoc "BibTeX" TeX-command-list)) TeX-command-list))
% eval: (setq TeX-command-list  (assq-delete-all (car (assoc "Biber"  TeX-command-list)) TeX-command-list))
% eval: (add-to-list 'TeX-command-list '("BibTeX" "bibtex LaTeX/%s" TeX-run-BibTeX nil t                                                                              :help "Run BibTeX") t)
% eval: (add-to-list 'TeX-command-list '("BibTeX" "bibtex LaTeX/%s" TeX-run-BibTeX nil (plain-tex-mode latex-mode doctex-mode ams-tex-mode texinfo-mode context-mode) :help "Run BibTeX") t)
% TeX-PDF-mode: t
% TeX-file-line-error: t
% TeX-debug-warnings: t
% LaTeX-command-style: (("" "%(PDF)%(latex) %(file-line-error) %(extraopts) -output-directory=LaTeX %S%(PDFout)"))
% TeX-source-correlate-mode: t
% TeX-parse-self: t
% eval: (cond ((string-equal system-type "darwin")    (progn (setq TeX-view-program-list '(("Skim" "/Applications/Skim.app/Contents/SharedSupport/displayline -b %n LaTeX/%o %b"))))))
% eval: (cond ((string-equal system-type "gnu/linux") (progn (setq TeX-view-program-list '(("Evince" "evince --page-index=%(outpage) LaTeX/%o"))))))
% eval: (cond ((string-equal system-type "gnu/linux") (progn (setq TeX-view-program-selection '((output-pdf "Evince"))))))
% TeX-parse-all-errors: t
% End:
